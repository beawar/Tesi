%----------------------------------------------------------------------------------------
%	MARGIN SETTINGS
%----------------------------------------------------------------------------------------

\geometry{
	paper=a4paper, % Change to letterpaper for US letter
	inner=2.5cm, % Inner margin
	outer=3.8cm, % Outer margin
	bindingoffset=.5cm, % Binding offset
	top=1.5cm, % Top margin
	bottom=1.5cm, % Bottom margin
%	showframe, % Uncomment to show how the type block is set on the page
}




% Genera automaticamente la pagina di copertina
%\newcommand{\makeFrontPage}{
%  % Declare new goemetry for the title page only.
%  \newgeometry{top=1cm}
%  
%  \begin{titlepage}
%  \begin{center}
%
%  \begin{center}
%  \includegraphics[width=8cm]{../../modello/or-bit_bkg.png}
%  \end{center}
%  
%%  \vspace{1cm}
%
%  \begin{Huge}
%  \textbf{\DocTitle{}}
%  \end{Huge}
%  
%  \textbf{\emph{Gruppo \GroupName{} \, \texttwelveudash{} \, Progetto \ProjectName{}}}
%  
%  \vspace{10pt}
%
%  \bgroup
%  \def\arraystretch{1.3}
%  \begin{tabular}{ r|l }
%    \multicolumn{2}{c}{\textbf{Informazioni sul documento}} \\
%    \hline
%		% differenzia a seconda che \DocVersion{} stampi testo o no
%		\setbox0=\hbox{\DocVersion{}\unskip}\ifdim\wd0=0pt
%			% nulla (non ho trovato come togliere l'a capo)
%			\\
%		\else
%			\textbf{Versione} & \DocVersion{} \\
%		\fi
%    \textbf{Redazione} & \multiLineCell[t]{\DocRedazione{}} \\
%    \textbf{Verifica} & \multiLineCell[t]{\DocVerifica{}} \\
%    \textbf{Approvazione} & \multiLineCell[t]{\DocApprovazione{}} \\
%    \textbf{Uso} & \DocUso{} \\
%    \textbf{Distribuzione} & \multiLineCell[t]{\DocDistribuzione{}} \\
%  \end{tabular}
%  \egroup
%
%  \vspace{10pt}
%
%  \textbf{Descrizione} \\
%  \DocDescription{}  
%
%%  \vspace{0.2cm}
%  
%
%  \end{center}
%  \begin{figure}[H]
%	\includegraphics[height=3cm, right]{../../modello/monolith}
%  \end{figure}
%  \end{titlepage}
%  
%  % Ends the declared geometry for the titlepage
%  \restoregeometry
%}

% -------------------------------------------------------
%  PAGINE INTERNE
% -------------------------------------------------------
%\fancypagestyle{plain}{	
%	% cancella tutti i campi di intestazione e piè di pagina
%	\fancyhf{}
%	\lhead{
%		\includegraphics[height=1.5cm, width=1.5cm, keepaspectratio=true]{../../modello/or-bit_bkg.png}
%		\parbox[b]{10cm}{
%			\emph{\GroupName{}} \vspace{0pt} \\
%			\emph{Progetto \ProjectName{}} \vspace{7pt}
%		}
%	}
%	\chead{}
%	%\rhead{
%	%	\slshape \leftmark
%	%}
%	% Stampa la sezione in alto a destra sull'header
%	\rhead{	
%		\includegraphics[height=1.25cm, width=1.25cm, keepaspectratio=true]{../../modello/monolith}
%	}
%	
%	\lfoot{
%		\DocTitle{} \\
%		% differenzia a seconda che \DocVersion{} stampi testo o no
%		\setbox0=\hbox{\DocVersion{}\unskip}\ifdim\wd0=0pt
%		% nulla
%		\else
%		v \DocVersion{}
%		\fi
%	}
%	\rfoot{Pagina \thepage{} di \pageref{LastPage}}
%	
%	% Visualizza una linea orizzontale in cima e in fondo alla pagina
%	\renewcommand{\headrulewidth}{0.3pt}
%	\renewcommand{\footrulewidth}{0.3pt}
%}
\setlength{\headheight}{30pt}
\pagestyle{plain}

% Per inserire del codice sorgente formattato
\usepackage{listings}
\definecolor{darkgray}{rgb}{.4,.4,.4}
\definecolor{purple}{rgb}{0.65, 0.12, 0.82}
\definecolor{mygreen}{rgb}{0,0.6,0}
\definecolor{mygray}{rgb}{0.5,0.5,0.5}
\definecolor{mymauve}{rgb}{0.58,0,0.82}

\lstset{
	extendedchars=true,          % lets you use non-ASCII characters
	inputencoding=utf8,   % converte i caratteri utf8 in latin1, richiede \usepackage{listingsutf8} anzichè listings
	basicstyle=\ttfamily,        % the size of the fonts that are used for the code
	breakatwhitespace=false,     % sets if automatic breaks should only happen at whitespace
	breaklines=true,             % sets automatic line breaking
	captionpos=b,                % sets the caption-position to top
	commentstyle=\color{mygreen},   % comment style
	frame=none,               % adds a frame around the code
	keepspaces=true,            % keeps spaces in text, useful for keeping indentation of code (possibly needs columns=flexible)
	keywordstyle=\color{blue}\bfseries,     % keyword style
	numbers=none,               % where to put the line-numbers; possible values are (none, left, right)
	numbersep=5pt,              % how far the line-numbers are from the code
	numberstyle=\color{mygray}, % the style that is used for the line-numbers
	rulecolor=\color{black},    % if not set, the frame-color may be changed on line-breaks within not-black text (e.g. comments (green here))
	showspaces=false,           % show spaces everywhere adding particular underscores; it overrides 'showstringspaces'
	showstringspaces=false,     % underline spaces within strings only
	showtabs=false,             % show tabs within strings adding particular underscores
	stepnumber=5,               % the step between two line-numbers. If it's 1, each line will be numbered
	stringstyle=\color{red},    % string literal style
	tabsize=4,                  % sets default tabsize
	firstnumber=1      % visualizza i numeri dalla prima linea
}

