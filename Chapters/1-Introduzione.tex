\chapter{Introduzione}
\begin{flushright}
	In questo capitolo introduttivo vengono presentati il contesto aziendale in cui si è svolto lo stage, il dominio applicativo analizzato e le premesse dello stage stesso.
\end{flushright}

\section{L'azienda}
Sanmarco Informatica nasce negli anni '80 come \gls{software house} specializzata nello sviluppo di applicazioni gestionali per aziende manifatturiere ed è oggi una leading company italiana nella progettazione e realizzazione di soluzioni a supporto della riorganizzazione di vari processi aziendali e professionali. L’ambizione e la volontà di rinnovarsi hanno permesso all'azienda di evolversi attraverso esperienze e scelte imprenditoriali di successo, che individuano nella specializzazione del proprio capitale umano l'elemento centrale. L'azienda, partner di \acrshort{IBM} Italia, cresce grazie all'impegno di 320 persone fra dipendenti e collaboratori, 13 distributori e 4 sedi – Grisignano di Zocco (VI), Reggio Emilia (RE), Tavagnacco (UD), Vimercate (MB). Le business units Sanmarco Informatica attive sono 5:
\begin{itemize}
	\item \textbf{Jgalileo}: soluzione \acrshort{ERP} per diverse tipologie di aziende manifatturiere. Jgalileo costituisce il prodotto di punta dell'azienda, diventando sempre più all'avanguardia grazie all'impegno di oltre 80 consulenti del centro di ricerca e sviluppo software. Questa soluzione gestionale copre le esigenze dell'intero processo aziendale attraverso eccellenze applicative completamente integrate ed è progettato su misura per interpretare al meglio le tipicità di ogni mercato.
	
	\item \textbf{4words}: web apps marketing solutions. La web agency sviluppa app \gls{IOS} e \Gls{android} per attività di prevendita, cataloghi prodotti, raccolta ordini, assistenza tecnica, al fine di offrire strumenti sempre più efficaci per la gestione di processi aziendali ERP in mobilità. 4words guida anche il cliente all'utilizzo della piattaforma web a supporto di strategie commerciali \acrshort{B2B} e \acrshort{B2C}.
	
	\item \textbf{NextBI}: analizza i dati dei clienti per ottimizzare le strategie di marketing. NextBI vanta infatti grande esperienza in progetti di \gls{CI}, \Gls{BI} e \Gls{CPM}, fornendo moduli già pronti e trasversali a tutti i contesti aziendali.
	
	\item \textbf{Discovery Quality}: ambito gestione documentale, qualità, sicurezza, ambiente. Questa suite per la gestione della Qualità, leader di mercato, garantisce al management tutti gli elementi necessari per gestire e misurare la Qualità e le prestazioni, gestendo efficacemente le istanze interne ed esterne di natura etica e sociale.
	
	\item \textbf{SMItech}: è il Team dedicato alla Tecnologia. L'obiettivo è aiutare i clienti a migliorare sicurezza ed efficienza informatica, attraverso la realizzazione di progetti di infrastruttura IT e lo sviluppo di servizi gestiti di \Gls{cybersecurity}.
\end{itemize}
Sanmarco Informatica è la prima ed unica azienda italiana entrata a far parte dell'\Gls{Open Power Foundation IBM}; a gennaio 2016 l'azienda ha ricevuto il riconoscimento internazionale \Gls{Beacon Award} come finalisti a livello mondiale fra le aziende d'eccellenza che propongono soluzioni tecnologiche innovative in combinazione con il sistema \Gls{Power} di IBM.

\subsection{4words}
4words è una web agency che sviluppa siti web per il business, curandone la realizzazione (web design e criteri di usabilità) e le attività di web – marketing (posizionamento \Gls{SEO}, campagne \Gls{PPC}/AdWords, social media marketing, annunci AdSense, campagne video).
4words realizza inoltre applicazioni per tablet relative al catalogo prodotti, orientate alla vendita, integrabili con siti B2B e B2C e capaci di creare un sistema integrato tra siti e software gestionale Erp.
La capacità di sviluppare applicazioni mobile integrabili con Sistemi Gestionali Erp rende 4words partner ideale per \Gls{PMI} e grandi aziende che vogliano fare un salto tecnologico significativo atto a cogliere potenzialità e vantaggi competitivi del web 2.0 e della connettività mobile.

\paragraph{Realizzazione siti web aziendali}: 4words cura la realizzazione e lo sviluppo di siti web aziendali, siti vetrina e catalogo prodotti al fine di far conoscere le realtà commerciali italiane su Internet. La realizzazione di un sito inizia dalla fase di pianificazione del progetto e della strategia da adottare per renderlo user – friendly e facilmente rintracciabile dai motori di ricerca. Successivamente, si passa alla cura della grafica e del design delle pagine. 4words è molto attenta a sviluppare siti web accattivanti e dai contenuti interessanti, al fine di migliorare il business e soddisfare i bisogni del cliente.

\paragraph{Realizzazione siti e-commerce B2C}: 4words si occupa della creazione di siti e-commerce per il tuo business realizzando shop online per la vendita di prodotti, basati su ampi database e configuratori semplici da utilizzare e ottimi per visionare e confrontare tutti gli articoli presenti sul sito. Il nostro team cura la realizzazione di e-commerce particolarmente efficaci sia dal punto di vista grafico sia dell'usabilità, affinché l'utente possa consultare rapidamente i prodotti e giungere successivamente al carrello, dove potrà concludere con semplicità l'acquisto.

\paragraph{Realizzazione portali}: 4words sviluppa portali B2C atti a contenere grandi quantità di informazioni e ideali per la raccolta dati relativi all'azienda (notizie su blog e forum, documenti consultabili e contenuti scaricabili dall'utente). I portali si rivolgono a tutti gli \gls{stakeholder}, siano essi fornitori, dipendenti o clienti dell'azienda e sono molto utili per i processi decisionali interni all'impresa.

\paragraph{Realizzazione siti business B2B}: La'azienda crea e sviluppa siti business B2B, ideali per chi vuole far conoscere la propria azienda su Internet e sviluppare la sua reputazione, grazie a una corretta visibilità sul web.
L'idea è quella di rendere visibili sia le PMI sia le grandi imprese, per realizzare nuove opportunità di business ed essere competitivi sul mercato. 4words adotta le migliori strategie affinché un'azienda sia facilmente rintracciabile sui motori di ricerca e sviluppi fedeltà nel cliente. Un cliente soddisfatto è un cliente che ritorna.