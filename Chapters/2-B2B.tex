\chapter{Il B2B}
\begin{flushright}
	\parbox{13cm}{\small In questo capitolo viene descritto e analizzato il portale B2B, sia a livello generale che nello specifico del contesto aziendale in cui si è svolto lo stage. Vengono presentati quindi vantaggi e svantaggi degli strumenti e delle tecnologie utilizzate.}
\end{flushright}
\section{Il B2B in generale}
Con B2B si intende un modello per la vendita di prodotti e servizi ad altre aziende. Un portale B2B può essere considerato come un'impresa di supporto che offre ciò di cui le altre aziende necessitano per vincere la concorrenza. Il B2B, infatti, offre prodotti grezzi, servizi o articoli in stock con prezzi molto vantaggiosi rispetto alla usuale vendita al pubblico, in quanto venduti il più delle volte direttamente dalla compagnia che li produce. Al contrario, nei portali B2C destinati al cliente singolo, il prodotto finito viene venduto con un prezzo che rispecchia tutte le piccole transazioni avvenute per la composizione dello stesso: l'azienda A che produce automobili ha dovuto acquistare i bulloni dall'azienda B, le vernici dall'azienda C e il vetro dei finestrini dall'azienda D. Tutti questi acquisti da terzi aumentano di fatto il costo dell'auto finale, che dovrà di per sè garantire la copertura delle spese ed un guadagno per il venditore.

Come qualsiasi altra attività commerciale, il modello B2B richiede una attenta pianificazione. Esso tipicamente fa affidamento su un rapporto solido che il team di vendita instaura con il cliente, mentre quanto inerente la promozione commerciale può includere pubblicità in riviste commerciali, \textit{convention} e conferenze, \textit{marketing} digitale (pubblicità online, tecniche SEO, newsletter) ed altre tecniche di sensibilizzazione tradizionali.

\subsection{Il B2B come e-commerce}
Con la diffusione del web e la rivoluzione digitale, è emerso un nuovo settore definito B2B \textit{e-commerce}. Tramite portali online, le compagnie hanno iniziato a vendere direttamente ad altre aziende, così come a condividere i dati e informazioni riguardo ai prodotti ed ai servizi in modo facile e rapido. Un portale web è infatti sempre raggiungibile e permettete una diffusione immediata
delle informazioni.

Parlando di B2B \textit{e-commerce}, possono essere individuate tre categorie principali:
\begin{description}
	\item[Siti web] Molte aziende necessitano di raggiungere specificatamente altre aziende e i loro impiegati. Il sito web può rappresentare il punto d'entrata di una rete esclusiva riservata ai clienti o agli utenti registrati, o di una rete interna per l'azienda stessa. Le attività possono inoltre vendere direttamente dal sito prodotti che di fatto non richiedono di essere visti dal vivo o provati prima dell'acquisto.
	
	Un esempio sono i B2B che vendono a loro volta B2B, componibili direttamente dal cliente, tramite strumenti di composizione, template, accesso al database, metodologie per le \textit{best practice} e integrazioni per il pagamento.
	
	\item[Fornitura e offerta] Conosciti come \textit{e-procurement}, questi siti sono di solito destinati ad un mercato di nicchia. Un agente può acquistare materiale dal fornitore, richiedere una proposta di vendita ed anche effettuare offerte per comprare ad un prezzo specifico.
	
	I portali delle industrie specializzate o a sviluppo verticale forniscono una sottorete informativa per la specifica industria, come ad esempio le industrie per la sanità, di costruzioni o di altri mercati specifici.
	I portali verticali hanno uno scopo più ampio rispetto ai siti web tradizionali, ma supportano comunque la vendita.
	
	I siti di intermediazione soddisfano i bisogni delle aziende per la fornitura e la richiesta in altro modo: questi siti agiscono come intermediari tra chi fornisce il servizio e il cliente potenziale.
	
	\item[Infomediari] La categoria finale è per i siti informativi, o \virgolette{infomediari}, che forniscono informazioni specializzate per specifiche aziende. Questi portali sono spesso usati come siti organizzativi per gli standard commerciali e industriali.
\end{description}
\section{Il B2B di 4words}
Descrizione e analisi del B2B di Sanmarco Informatica.
\subsection{Struttura del progetto}
Come è strutturato il B2B, sia a livello funzionale, sia a livello progettuale
\subsection{Versionamento}
Come viene versionato
\subsection{Tecnologie utilizzate}
Analisi delle tecnologie utilizzate
\subsubsection{Back-end}
\subsubsection{Front-end}