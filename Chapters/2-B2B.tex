\chapter{Il B2B}
\begin{flushright}
	\parbox{13cm}{\small In questo capitolo viene descritto e analizzato il portale B2B, sia a livello generale che nello specifico del contesto aziendale in cui si è svolto lo stage. Vengono analizzate le funzionalità, i fattori di successo e le tecnologie utilizzate, presentandone vantaggi e svantaggi.}
\end{flushright}
\section{Il B2B in generale}
Con B2B si intende un modello per la vendita di prodotti e servizi ad altre aziende. Un portale B2B può essere considerato come un'impresa di supporto che offre ciò di cui le altre aziende necessitano per vincere la concorrenza. Il B2B, infatti, offre prodotti grezzi, servizi o articoli in stock con prezzi molto vantaggiosi rispetto alla usuale vendita al pubblico, in quanto venduti il più delle volte direttamente dalla compagnia che li produce. Al contrario, nei portali B2C destinati al cliente singolo, il prodotto finito viene venduto con un prezzo che rispecchia tutte le piccole transazioni avvenute per la composizione dello stesso: l'azienda A che produce automobili ha dovuto acquistare i bulloni dall'azienda B, le vernici dall'azienda C e il vetro dei finestrini dall'azienda D. Tutti questi acquisti da terzi aumentano di fatto il costo dell'auto finale, che dovrà di per sè garantire la copertura delle spese ed un guadagno per il venditore.

Come qualsiasi altra attività commerciale, il modello B2B richiede una attenta pianificazione. Esso tipicamente fa affidamento su un rapporto solido che il team di vendita instaura con il cliente, mentre quanto inerente la promozione commerciale può includere pubblicità in riviste commerciali, \textit{convention} e conferenze, \textit{marketing} digitale (pubblicità online, tecniche SEO, newsletter) ed altre tecniche di sensibilizzazione tradizionali.

\subsection{Il B2B come e-commerce}
Con la diffusione del web e la rivoluzione digitale, è emerso un nuovo settore definito B2B \textit{e-commerce}. Tramite portali online, le compagnie hanno iniziato a vendere direttamente ad altre aziende, così come a condividere i dati e informazioni riguardo ai prodotti ed ai servizi in modo facile e rapido. Un portale web è infatti sempre raggiungibile e permettete una diffusione immediata
delle informazioni.

Parlando di B2B \textit{e-commerce}, possono essere individuate tre categorie principali:
\begin{description}
	\item[Siti web] Molte aziende necessitano di raggiungere specificatamente altre aziende e i loro impiegati. Il sito web può rappresentare il punto d'entrata di una rete esclusiva riservata ai clienti o agli utenti registrati, o di una rete interna per l'azienda stessa. Le attività possono inoltre vendere direttamente dal sito prodotti che di fatto non richiedono di essere visti dal vivo o provati prima dell'acquisto.
	
	Un esempio sono i B2B che vendono a loro volta B2B, componibili direttamente dal cliente, tramite strumenti di composizione, template, accesso al database, metodologie per le \textit{best practice} e integrazioni per il pagamento.
	
	\item[Fornitura e offerta] Conosciti come \textit{e-procurement}, questi siti sono di solito destinati ad un mercato di nicchia. Un agente può acquistare materiale dal fornitore, richiedere una proposta di vendita ed anche effettuare offerte per comprare ad un prezzo specifico.
	
	I portali delle industrie specializzate o a sviluppo verticale forniscono una sottorete informativa per la specifica industria, come ad esempio le industrie per la sanità, di costruzioni o di altri mercati specifici.
	I portali verticali hanno uno scopo più ampio rispetto ai siti web tradizionali, ma supportano comunque la vendita.
	
	I siti di intermediazione soddisfano i bisogni delle aziende per la fornitura e la richiesta in altro modo: questi siti agiscono come intermediari tra chi fornisce il servizio e il cliente potenziale.
	
	\item[Infomediari] La categoria finale è per i siti informativi, o \virgolette{infomediari}, che forniscono informazioni specializzate per specifiche aziende. Questi portali sono spesso usati come siti organizzativi per gli standard commerciali e industriali.
\end{description}

\subsubsection{Funzionalità base}
Per essere un buon portale, un B2B deve includere alcune funzionalità fondamentali che ne determinano il successo. Queste comprendono ad esempio la gestione dell'ordine, la visualizzazione dei prodotti e la gestione dei clienti. Vi sono poi aspetti legati all'usabilità, come la semplicità nella creazione dell'ordine e l'efficienza nella ricerca di elementi come prodotti o clienti, ed altri inerenti le tecniche SEO.

\paragraph{L'ordine}
L'ordine è l'elemento alla base del B2B. Lo scopo primario di gran parte dei portali B2B è semplificare i processi interaziendali di fornitura e acquisto. L'ordine è anche l'entità più complessa che il B2B gestisce: vi sono infatti moltissimi fattori che dipendono non solo dall'azienda che ne dispone, ma anche dalle leggi statali e continentali in vigore, che possono cambiare anche frequentemente. Si pensi ad esempio all'\Gls{IVA}, una tassa applicata solo in alcuni Paesi, con regole e percentuali differenti. Un buon B2B deve essere in grado, soprattutto se destinato ad un mercato internazionale, o comunque in espansione, di gestire tutte queste variabili, a seconda di dove sta chi vende e chi compra. Non a caso una azienda per la gestione commerciale-amministrativa degli ordini necessita e si avvale di software gestionali, creati appositamente per questo scopo. Il B2B deve quindi essere in grado di riportare le logiche del gestionale nel web, rispettando le convenzioni di quest'ultimo e proponendo delle operazioni \virgolette{guidate}, in modo tale che il sistema sia utilizzabile anche da chi di gestionale ne sa poco o nulla.

\paragraph{Il catalogo}
I prodotti devono essere reperibili ed inseribili in un'ordine. Il loro dettagli deve essere chiaro: un prodotto è sempre caratterizzato da un codice che lo identifica in modo univoco, e che viene spesso utilizzato per acquisti \virgolette{rapidi}, soprattutto quando vengono effettuati ordini ripetuti (un'azienda che utilizza per tutti i macchinari un certo bullone inserirà nell'ordine direttamente il codice, piuttosto che ricercare il prodotto tramite il nome \virgolette{Bullone con testa esagonale m8}).

\paragraph{Caratteristiche generali degli e-commerce}
Essendo un portale destinato agli acquisti, il B2B deve avere le funzionalità base di questa categoria: un carrello per poter raggruppare i prodotti che si vogliono comprare e sapere in anticipo il prezzo totale; un pannello di controllo per il cliente (la \textit{dashboard}) per controllare gli ultimi movimenti e il loro stato; l'accesso al profilo per la modifica di informazioni personali; la possibilità di effettuare il login, fondamentale per poter tenere traccia delle attività svolte dagli utenti. Quest'ultimo fattore svolge un ruolo importante per determinare la strategia di marketing da parte dell'amministrazione.

\paragraph{Filtri e ricerca}
La ricerca di elementi all'interno del portale è una funzionalità che non può mancare, soprattutto con un numero di elementi molto elevato, come spesso avviene nei B2B. Allo stesso modo applicare dei filtri riduce i tempi che l'utente impiega per ottenere ciò che vuole, aumentando di fatto la sua soddisfazione.

\paragraph{Altre funzionalità}
Altre funzionalità che si possono definire \textit{must-have} sono:
\begin{itemize}
	\item configurazione dell'ordine;
	\item informazioni sulla disponibilità e sulla consegna;
	\item prezzo del prodotto pensato sul cliente;
	\item sconti e promozioni;
	\item pagamenti sicuri ed in varie modalità;
	\item \textit{Search Engine Optimization};
\end{itemize}
È necessario inoltre che siano supportati tutti i browser per le versioni più recenti e che le performance del sistema siano buone, sia in termini di caricamento delle pagine, sia per quanto riguarda i tempi di esecuzione di \textit{query} di ricerca nel database.

\subsubsection{I fattori di successo del B2B}
Il commercio \textit{business-to-business} è sempre stato uno dei vantaggi principali per la predominanza delle aziende nella competizione internazionale. L'avvento di Internet e le nuove modalità di commercio e comunicazione hanno però modificato profondamente le dinamiche e i processi tradizionali per tale mercato. Da alcune ricerche sono quindi emersi i fattori critici che una azienda deve considerare per avere successo nel B2B del web: 21 elementi, suddivisi in 5 categorie (strategia di marketing, sito web, dimensione globale, fattori interni e fattori esterni). Per individuarli sono state usate alcune tecniche specifiche, come
\begin{itemize}
	\item scansione ambientale;
	\item analisi della struttura industriale;
	\item opinione di esperti del settore;
	\item analisi della concorrenza;
	\item \textit{best practice};
	\item valutazione interna;
	\item fattori di intuizione;
	\item analisi del profitto di strategie commerciali.
\end{itemize}
Per quanto invece riguarda i fattori proposti, la forza vendita gioca un ruolo centrale nello sviluppo delle strategie di mercato se viene fornita una formazione appropriata. Il coinvolgimento di fornitori e clienti, la cultura del web e l'utilizzo di entrambi i mezzi (tradizionale ed online) sono altri fattori importanti, così come la sicurezza, la fiducia e la confidenza tra il venditore ed il potenziale cliente. L'\textit{e-commerce} è come una relazione tra le parti, dove le alleanze, l'organizzazione e le comunicazioni sono rese possibili dalle nuove tecnologie.

\paragraph{Strategia di marketing}
\subparagraph{Supporto e impegno manageriale}
È richiesta la conoscenza delle potenzialità di Internet da parte del manager, che ha anche il compito di diffonderle in modo proattivo all'interno dell'azienda. L'impegno per il commercio via web aiuta a promuovere il suo sviluppo anche in altre aziende, ma richiede comunque supporto finanziario: c'è un'importante correlazione tra l'investimento effettuato ed il guadagno ottenuto. Per questo il coinvolgimento del livello amministrativo gioca il ruolo più critico.

\subparagraph{Obiettivi strategici}
Il successo dello sviluppo del web B2B dipendono da quanto chiaramente sono definiti gli obiettivi strategici per una organizzazione.

\subparagraph{Integrazione tra Internet e la strategia di marketing}
I responsabili marketing web di successo sono quelli che costruiscono un sistema in grado di integrarsi con le applicazioni esistenti e che possano offrire la formazione necessaria per il suo utilizzo. Sebbene Internet offre molti dei servizi, esso non rimpiazzerà i mezzi tradizionali: i clienti che comprano online continuano comunque a comprare attraverso altri mezzi. Pertanto, le aziende devono considerare il web-marketing come un complemento, piuttosto che un rimpiazzo. Molti clienti infatti preferiscono valutare un prodotto online e poi procedere con l'acquisto in altri modi, di persona o al telefono ad esempio.

\subparagraph{Definizione del target}
Definire chi tra clienti esterni, fornitori, venditori, rivenditori ed altri \textit{business partner} costituisce il target è una operazione primaria, in quanto determina come i canali, gli strumenti e il target stesso sono utilizzati e coinvolti.

\paragraph{Sito web}
\subparagraph{Design}
Un sito web ben progettato e con un bel design è il biglietto da visita dell'azienda nel web. La creazione di un portale richiede però un continuo sforzo per il suo aggiornamento e mantenimento, per continuare a soddisfare le aspettative degli utenti in base alle tendenze. Un ulteriore fattore è il contenuto, che deve essere di valore, accurato ed aggiornato, sia per attrarre nuovi clienti da ogni dove, sia per incoraggiarli a ritornare; le informazioni devono essere chiare e consistenti, in quanto i clienti fanno sempre una valutazione sulla loro utilità nel condurlo a prendere decisioni. La maggior parte degli utenti arriva sul web in cerca di informazioni, pertanto un portale che offre più dettagli sull'azienda e sui prodotti avrà sicuramente più successo. D'altro canto, le performance per l'accesso a tali informazioni sono un dato altrettanto importante: la progettazione del catalogo dei prodotti va sviluppata ancor prima dello stesso portale, in quanto determina la velocità di esecuzione delle \textit{query} di ricerca per migliaia di prodotti.

Vanno quindi considerate tutte le regole base di usabilità, tra cui la semplificazione della navigazione con \gls{breadcrumb} e \gls{tag}, l'interazione e la reattività verso i \gls{feedback} degli utenti, lo scambio di informazioni tra utenti e l'integrazione del web con altri canali di marketing.\virgolette{IL sito web non deve servire solamente come interfaccia di raccolta ordini ma ha anche un alto valore aggiunto per il contenuto informativo}\autocite{b2bSuccessFactors}

\subparagraph{Promozione (offline ed online)}
La promozione è importante per due motivi: il proprio sito web deve essere riconoscibile rispetto a quello della competizione. Questo richiede una alta accessibilità, cosicché sia il sito che la sua pubblicità possano raggiungere il maggior numero di utenti possibile. Secondo, la promozione di un sito web  richiede le conoscenze tecniche di un esperto che sappia come l'utente medio trova usualmente i contenuti su Internet.

\paragraph{Fattori globali}
\subparagraph{Comprensione dell'ambiente esterno}
Un sito dovrebbe rispettare il sistema degli stati in cui è utilizzato. Vanno quindi studiati gli ambienti, incluse le regolazioni del commercio e le modalità di spedizione, per comprendere i vantaggi dei prodotti e dei servizi locali. Il mercato internazionale richiede che vengano effettuate molte considerazioni gestionali e di pianificazione, tra cui lo standard dei prodotti locali, i prezzi di mercato, i fattori competitivi, la valuta, i problemi delle modalità di pagamento, l'assistenza e i servizi offerti ai clienti e considerazioni riguardati le leggi.

\subparagraph{Risorse}
Per essere pronti all'incremento delle vendite che il portale web può portare, sono necessarie risorse, che non tutte le aziende potrebbero avere. Alcuni esempi sono la possibilità di ricevere ordini 24 ore su 24, un servizio clienti efficiente e l'esperienza per la gestione delle spedizioni internazionali

\subparagraph{Multilingua}
Uno dei problemi principali per la comunicazione a livello globale è la lingua. Per le compagnie che vogliono utilizzare il web a livello internazionale, la lingua è la sfida da superare. Per il commercio internazionale è necessario fornire una traduzione per almeno le lingue base, in modo tale da ridurre le difficoltà di comprensione per i lettori non madre-lingua.

\subparagraph{Considerazioni culturali}
Così come la lingua, anche gli altri fattori culturali devono essere riportati nel portale web. Inoltre, fornire informazioni che possano risultare interessanti da una varietà di persone con bisogni e gusti diversi può incoraggiare l'incontro tra le differenti nazioni e culture. Va considerato che anche gli elementi non testuali possono causare incomprensioni culturali, come per esempio l'uso dei colori o dei simboli delle icone (il colore bianco, che in gran parte del mondo è neutrale o addirittura elegante, in alcuni Paesi dell'Asia significa morte).

\subparagraph{Consegna internazionale}
Progettare un sistema logistico in grado di effettuare consegne in più nazioni in modo efficiente è necessario se si vuole vendere in più stati e favorisce la scalabilità per le medie aziende in espansione. Come minimo, quindi, il portale B2B dovrebbe indicare i costi e le tempistiche di spedizione verso tutti i Paesi in cui è disponibile la consegna.

\paragraph{Fattori interni}
\subparagraph{Infrastruttura tecnologica}
Affinché tutto il sistema di cui si è parlato fino ad ora funzioni correttamente, è necessario che l'infrastruttura tecnologica alla base offra buone prestazioni. Questo non è comunque l'unico aspetto da considerare, in quanto Internet per essere uno strumento utile ha bisogno che i suoi utilizzatori abbiano familiarità con il PC e sappiano apprezzare i benefici e le potenzialità che offre.

\subparagraph{Cultura interna}
Un'azienda deve comprendere e abbracciare i nuovi valori, i processi di gestione e gli stili di comunicazione che i nuovi modi di fare marketing creano. Iniziare a commerciare via web è come cominciare in un nuovo stato: la chiave del successo sta nel comprendere, apprezzare e onorare la cultura ed i protocolli di quel paese.

\subparagraph{Forza vendita}
La forza vendita ha un ruolo centrale nel successo di un portale B2B, in quanto in grado di migliorare la comunicazione, aiutando il cliente ad approcciarsi B2B e integrando nuovi strumenti di gestione dell'informazione. Le conoscenze degli addetti al marketing, inoltre, rimangono fondamentali, dato che un sito per quanto tecnologicamente avanzato non può aver successo se non rispetta le aspettative del cliente.

\subparagraph{Formazione}
Una adeguata formazione risulta importante per mantenere aggiornati gli utenti, siano essi interni od esterni, sulle nuove funzionalità e potenzialità delle tecnologie e del sistema.

\paragraph{Fattori esterni}
\subparagraph{Fiducia}
Più l'utente è cosciente di quello che sta facendo nel web, più vorrà delle garanzie. La fiducia verso un sito web sta quindi diventando la chiave che determinerà il successo o il fallimento di molte compagnie. La fiducia è ormai più importante nel monto virtuale che in quello reale, poiché le parti coinvolte in una operazione come può essere l'ordine non sono più nello stesso luogo e di certo non possono dipendere su variabili come la prossimità fisica, la stretta di mano o i segnali del corpo.

\subparagraph{Sicurezza}
Una delle preoccupazioni più diffuse nel web è la sicurezza delle transazioni finanziarie, tanto che alcuni preferiscono guardare i prodotti online, per poi utilizzare metodi offline per confermare l'ordine, come il telefono.

\subparagraph{Relazioni}
I continui cambiamenti che avvengono nel mondo dell'informatica hanno determinato un cambiamento nelle relazioni: esse si basano sempre più sull'informazione che può essere trasmessa tra aziende, piuttosto che sulle modalità con cui questa transazione avviene, come prevede invece la visione tradizionale. Svolgono un ruolo fondamentale quindi i contenuti, le tecnologie e le strategie di marketing.

\subparagraph{Infrastruttura di rete}
Uno dei fattori esterni che non dipendono direttamente dall'azienda è l'infrastruttura di rete: per sfruttare al massimo i vantaggi che il mercato virtuale offre, è necessaria una infrastruttura in grado di supportare le novità che il settore propone. 

\subparagraph{Coinvolgimento del cliente}
Come ultimo punto, ma non meno importante, è il coinvolgimento del cliente in questo tipo di attività. Le aziende dovrebbero impegnarsi a motivare i clienti ad effettuare il passaggio all'ambiente online, oltre che a predisporre un sistema interno in grado di rispondere rapidamente alle richieste dei clienti.
Il modello ottimale che soddisfa a pieno i clienti si basa infatti su una infrastruttura \textit{e-business} con quattro caratteristiche: è facile da usare, ha molte funzionalità, è affidabile ed offre prestazioni elevate.

\section{Il B2B di 4words}
Il B2B di Sanmarco Informatica è un prodotto completo ed al contempo in continua evoluzione. Svolge tutte le funzionalità base che un buon B2B dovrebbe avere e si integra pienamente con gli altri servizi dell'azienda, in particolar modo con il gestionale, agevolando così attività quali l'inserimento e la gestione degli di ordini, la gestione dei documenti e degli appuntamenti, la visualizzazione del catalogo, la configurazione di prodotti. Il B2B è fornito in varie versioni per adattarsi meglio alle esigenze dei clienti:
\begin{itemize}
	\item versione standard (il più diffuso);
	\item versione con web-services;
	\item versione moda.
\end{itemize}
In questa relazione viene trattata solamente la versione standard, l'unica su cui si è incentrato lo stage, mentre verrà solamente presentata la versione con web-services.

\subsection{Funzionalità principali}
In generale, il portale è composto da moduli componibili, in modo tale da essere di per sè un prodotto funzionante, ma anche facilmente personalizzabile secondo le necessità del cliente. I moduli principali sono quello dell'ordine, delle interrogazioni, l'area amministrativa, l'area dei documenti e l'agenda. Tutte queste aree sono attivabili o disattivabili per ogni utente, a seconda del suo gruppo di appartenenza. 

\subsubsection{Abilitazioni e permessi}
Una delle funzionalità principali è infatti la caratterizzazione degli utenti per ruoli. Ad ogni ruolo può essere associata un'abilitazione differente, in modo tale da adattare l'esperienza di navigazione per ognuno di essi. Si considerino per esempio gli amministratori, gli agenti di vendita e i clienti: sono tre tipi di utenza, ai quali però vanno mostrate aree differenti. Gli amministratori possono avere il controllo su tutto, in particolar modo sulle configurazioni. Sono in grado quindi di creare, modificare ed eliminare utenti, bloccarli o sbloccarli, gestire gli ordini, creare, modificare e rimuovere abilitazioni con i relativi permessi, e molto altro. L'amministratore di fatto ha il controllo completo del B2B, può arrivare a qualsiasi area, anche quelle a più basso livello, come la creazione delle connessioni ai database. L'agente di vendita potrebbe invece poter solamente creare ordini, modificandoli fin tanto che non sono inviati, creare e gestire i propri clienti, senza vedere quelli di altri agenti, così come per gli appuntamenti. La navigazione dell'agente è quindi totalmente focalizzata sulla sua funzione all'interno dell'azienda e come nel sistema tradizionale, egli non ha accesso a informazioni riguardanti suoi co-ruolo. Possono fare eccezione i capogruppo, ovvero agenti con un grado di controllo maggiore, che permette loro di controllare le attività dei subordinati. Il ruolo di cliente, infine, è generalmente quello con i permessi minori: il cliente di solito è un'altra azienda o organizzazione, che non deve quindi avere acceso ad informazioni riservate.\\
Le abilitazioni e i permessi sono di fatto del tutto personalizzabili senza l'ausilio di un tecnico 4words. Gli utenti sono inseribili, come visto, in gerarchie, che semplificano ancor di più la gestione da parte dell'amministratore, il quale in pochi passi è in grado di definire regole per molti utenti contemporaneamente. All'utilizzatore finale tutta questa serie di permessi è totalmente trasparente, in quanto non vede che esistono altre aree a lui inaccessibili.

\subsubsection{L'ordine}
Come per ogni B2B, l'ordine rappresenta il fulcro del portale e l'entità più complessa. Per rendere la creazione dell'ordine completa e al tempo stesso \textit{user-friendly}, essa è stata suddivisa in vari step che aiutano l'utente a inserire le informazioni richieste correttamente, presentandogli un numero ridotto di informazioni da fornire e, dove possibile, con aiuti, come ad esempio tramite la scelta multipla (e dunque limitata). Tramite questo \gls{wizard} è quindi possibile inviare ordini, ma anche iniziarli e lasciarli in sospeso per inviarli successivamente.

\subsubsection{Parcheggio}
Una volta inviato, l'ordine può essere inserito direttamente nel gestionale oppure ricevere dei blocchi (amministrativi o commerciali ad esempio), i quali determinano la necessità di approvazione da parte dell'ufficio relativo per il processamento dell'ordine stesso. Questo blocco è detto parcheggio e, con tutte le personalizzazioni possibili in termini di regole e modalità, è un'area molto utile in quanto velocizza alcune operazioni standard. Un tipico esempio di utilizzo è per i clienti fuori fido: a questa categoria viene permesso di fare ordini solamente a certe condizioni, verificate personalmente da alcuni addetti. Ecco quindi che il B2B permette a tali persone di accedere direttamente a tutti gli ordini sottoposti a controllo, sbloccandoli o rifiutandoli con un semplice click, con la possibilità di notificare l'utente di quanto deciso via email. Questa procedura semplifica molto operazioni che fatte direttamente nel gestionale potrebbero risultare non solo più difficoltose, ma anche meno immediate, in quanto per accedere al B2B è sufficiente un tablet o un pc connesso ad internet, mentre il gestionale è solitamente disponibile solamente dentro all'azienda.

\subsubsection{Interrogazioni}
L'area delle interrogazioni è dedicata alla ricerca di schede e informazioni riguardanti i dati inseriti dall'utente. Questi dati possono essere clienti o ordini, e sono suddivisi in varie categorie, così da permettere all'utente di trovare facilmente quello che cerca. In particolare, l'elenco dei clienti creati o utilizzati vengono raccolti in un unica videata, con tutte le loro informazioni. Anche per gli ordini esiste uno storico che indica i dettagli degli ordini inviati, bloccati o sospesi, dove è possibile applicare dei filtri di ricerca, velocizzando molto il procedimento che altrimenti sarebbe stato eseguito nel gestionale.

\subsubsection{Catalogo}
Il catalogo prodotti nella sua versione standard rispetta i requisiti del modello B2B di successo. Esso è presentato in varie modalità, secondo le convenzioni del web, tra cui la visualizzazione a lista (con più dettagli per ogni articolo) e quella a griglia, in cui hanno un ruolo importante le immagini. Questa seconda modalità è di solito rivolta a chi con il proprio B2B vuole rispecchiare gli e-commerce più famosi, come Amazon o Ebay, piuttosto che a clienti che vendono prodotti tecnici di cui sono più importanti le caratteristiche della forma. Il catalogo B2B non solo permette di effettuare ordini secondo le abitudini degli utenti (il 66\% degli utenti di Internet ha fatto acquisti online nel 2016\autocite{eurostat}), ma ha anche la funzione di sostituire i cataloghi stampati che gli agenti di vendita propongono al cliente, in quanto è legato all'utente autenticato ed è una fonte sempre aggiornata. Ad ogni prodotto è possibile associare la disponibilità, aggiornata in tempo reale rispetto al gestionale, riducendo così significativamente il numero di ordini con prodotti non disponibili o non più in produzione. Per l'agente questa è una caratteristica importante, in quanto non è più necessario mettersi in contatto con la sede centrale per conoscere cambiamenti effettuati ai prodotti, anche in termini di prezzi e listini.

\subsection{Struttura del progetto}
Come è strutturato il B2B, sia a livello funzionale, sia a livello progettuale
\subsection{Versionamento}
Come viene versionato
\subsection{Tecnologie utilizzate}
Analisi delle tecnologie utilizzate
\subsubsection{Back-end}
\subsubsection{Front-end}