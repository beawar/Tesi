\chapter{Il B2B}
\begin{flushright}
	\parbox{13cm}{\small In questo capitolo viene descritto e analizzato il portale B2B, sia a livello generale che nello specifico del contesto aziendale in cui si è svolto lo stage. Vengono presentati quindi vantaggi e svantaggi degli strumenti e delle tecnologie utilizzate.}
\end{flushright}
\section{Il B2B in generale}
Con B2B si intende un modello per la vendita di prodotti e servizi ad altre aziende. Un portale B2B può essere considerato come un'impresa di supporto che offre ciò di cui le altre aziende necessitano per vincere la concorrenza. Il B2B, infatti, offre prodotti grezzi, servizi o articoli in stock con prezzi molto vantaggiosi rispetto alla usuale vendita al pubblico, in quanto venduti il più delle volte direttamente dalla compagnia che li produce. Al contrario, nei portali B2C destinati al cliente singolo, il prodotto finito viene venduto con un prezzo che rispecchia tutte le piccole transazioni avvenute per la composizione dello stesso: l'azienda A che produce automobili ha dovuto acquistare i bulloni dall'azienda B, le vernici dall'azienda C e il vetro dei finestrini dall'azienda D. Tutti questi acquisti da terzi aumentano di fatto il costo dell'auto finale, che dovrà di per sè garantire la copertura delle spese ed un guadagno per il venditore.

Come qualsiasi altra attività commerciale, il modello B2B richiede una attenta pianificazione. Esso tipicamente fa affidamento su un rapporto solido che il team di vendita instaura con il cliente, mentre quanto inerente la promozione commerciale può includere pubblicità in riviste commerciali, \textit{convention} e conferenze, \textit{marketing} digitale (pubblicità online, tecniche SEO, newsletter) ed altre tecniche di sensibilizzazione tradizionali.

\subsection{Il B2B come e-commerce}
Con la diffusione del web e la rivoluzione digitale, è emerso un nuovo settore definito B2B \textit{e-commerce}. Tramite portali online, le compagnie hanno iniziato a vendere direttamente ad altre aziende, così come a condividere i dati e informazioni riguardo ai prodotti ed ai servizi in modo facile e rapido. Un portale web è infatti sempre raggiungibile e permettete una diffusione immediata
delle informazioni.

Parlando di B2B \textit{e-commerce}, possono essere individuate tre categorie principali:
\begin{description}
	\item[Siti web] Molte aziende necessitano di raggiungere specificatamente altre aziende e i loro impiegati. Il sito web può rappresentare il punto d'entrata di una rete esclusiva riservata ai clienti o agli utenti registrati, o di una rete interna per l'azienda stessa. Le attività possono inoltre vendere direttamente dal sito prodotti che di fatto non richiedono di essere visti dal vivo o provati prima dell'acquisto.
	
	Un esempio sono i B2B che vendono a loro volta B2B, componibili direttamente dal cliente, tramite strumenti di composizione, template, accesso al database, metodologie per le \textit{best practice} e integrazioni per il pagamento.
	
	\item[Fornitura e offerta] Conosciti come \textit{e-procurement}, questi siti sono di solito destinati ad un mercato di nicchia. Un agente può acquistare materiale dal fornitore, richiedere una proposta di vendita ed anche effettuare offerte per comprare ad un prezzo specifico.
	
	I portali delle industrie specializzate o a sviluppo verticale forniscono una sottorete informativa per la specifica industria, come ad esempio le industrie per la sanità, di costruzioni o di altri mercati specifici.
	I portali verticali hanno uno scopo più ampio rispetto ai siti web tradizionali, ma supportano comunque la vendita.
	
	I siti di intermediazione soddisfano i bisogni delle aziende per la fornitura e la richiesta in altro modo: questi siti agiscono come intermediari tra chi fornisce il servizio e il cliente potenziale.
	
	\item[Infomediari] La categoria finale è per i siti informativi, o \virgolette{infomediari}, che forniscono informazioni specializzate per specifiche aziende. Questi portali sono spesso usati come siti organizzativi per gli standard commerciali e industriali.
\end{description}

\subsubsection{Funzionalità base}
Per essere un buon portale, un B2B deve includere alcune funzionalità fondamentali che ne determinano il successo. Queste comprendono ad esempio la gestione dell'ordine, la visualizzazione dei prodotti e la gestione dei clienti. Vi sono poi aspetti legati all'usabilità, come la semplicità nella creazione dell'ordine e l'efficienza nella ricerca di elementi come prodotti o clienti, ed altri inerenti le tecniche SEO.

\paragraph{L'ordine}
L'ordine è l'elemento alla base del B2B. Lo scopo primario di gran parte dei portali B2B è semplificare i processi interaziendali di fornitura e acquisto. L'ordine è anche l'entità più complessa che il B2B gestisce: vi sono infatti moltissimi fattori che dipendono non solo dall'azienda che ne dispone, ma anche dalle leggi statali e continentali in vigore, che possono cambiare anche frequentemente. Si pensi ad esempio all'\Gls{IVA}, una tassa applicata solo in alcuni Paesi, con regole e percentuali differenti. Un buon B2B deve essere in grado, soprattutto se destinato ad un mercato internazionale, o comunque in espansione, di gestire tutte queste variabili, a seconda di dove sta chi vende e chi compra. Non a caso una azienda per la gestione commerciale-amministrativa degli ordini necessita e si avvale di software gestionali, creati appositamente per questo scopo. Il B2B deve quindi essere in grado di riportare le logiche del gestionale nel web, rispettando le convenzioni di quest'ultimo e proponendo delle operazioni \virgolette{guidate}, in modo tale che il sistema sia utilizzabile anche da chi di gestionale ne sa poco o nulla.

\paragraph{Il catalogo}
I prodotti devono essere reperibili ed inseribili in un'ordine. Il loro dettagli deve essere chiaro: un prodotto è sempre caratterizzato da un codice che lo identifica in modo univoco, e che viene spesso utilizzato per acquisti \virgolette{rapidi}, soprattutto quando vengono effettuati ordini ripetuti (un'azienda che utilizza per tutti i macchinari un certo bullone inserirà nell'ordine direttamente il codice, piuttosto che ricercare il prodotto tramite il nome \virgolette{Bullone con testa esagonale m8}).

\paragraph{Caratteristiche generali degli e-commerce}
Essendo un portale destinato agli acquisti, il B2B deve avere le funzionalità base di questa categoria: un carrello per poter raggruppare i prodotti che si vogliono comprare e sapere in anticipo il prezzo totale; un pannello di controllo per il cliente (la \textit{dashboard}) per controllare gli ultimi movimenti e il loro stato; l'accesso al profilo per la modifica di informazioni personali; la possibilità di effettuare il login, fondamentale per poter tenere traccia delle attività svolte dagli utenti. Quest'ultimo fattore svolge un ruolo importante per determinare la strategia di marketing da parte dell'amministrazione.

\paragraph{Filtri e ricerca}
La ricerca di elementi all'interno del portale è una funzionalità che non può mancare, soprattutto con un numero di elementi molto elevato, come spesso avviene nei B2B. Allo stesso modo applicare dei filtri riduce i tempi che l'utente impiega per ottenere ciò che vuole, aumentando di fatto la sua soddisfazione.

\paragraph{Altre funzionalità}
Altre funzionalità che si possono definire \textit{must-have} sono:
\begin{itemize}
	\item configurazione dell'ordine;
	\item informazioni sulla disponibilità e sulla consegna;
	\item prezzo del prodotto pensato sul cliente;
	\item sconti e promozioni;
	\item pagamenti sicuri ed in varie modalità;
	\item \textit{Search Engine Optimization};
\end{itemize}
È necessario inoltre che siano supportati tutti i browser per le versioni più recenti e che le performance del sistema siano buone, sia in termini di caricamento delle pagine, sia per quanto riguarda i tempi di esecuzione di \textit{query} di ricerca nel database.

\subsubsection{I fattori di successo del B2B}
Il commercio \textit{business-to-business} è sempre stato uno dei vantaggi principali per la predominanza delle aziende nella competizione internazionale. L'avvento di Internet e le nuove modalità di commercio e comunicazione hanno però modificato profondamente le dinamiche e i processi tradizionali per tale mercato. Da alcune ricerche sono quindi emersi i fattori critici che una azienda deve considerare per avere successo nel B2B del web: 21 elementi, suddivisi in 5 categorie (strategia di marketing, sito web, dimensione globale, fattori interni e fattori esterni). Per individuarli sono state usate alcune tecniche specifiche, come
\begin{itemize}
	\item scansione ambientale;
	\item analisi della struttura industriale;
	\item opinione di esperti del settore;
	\item analisi della concorrenza;
	\item \textit{best practice};
	\item valutazione interna;
	\item fattori di intuizione;
	\item analisi del profitto di strategie commerciali.
\end{itemize}
Per quanto invece riguarda i fattori proposti, la forza vendita gioca un ruolo centrale nello sviluppo delle strategie di mercato se viene fornita una formazione appropriata. Il coinvolgimento di fornitori e clienti, la cultura del web e l'utilizzo di entrambi i mezzi (tradizionale ed online) sono altri fattori importanti, così come la sicurezza, la fiducia e la confidenza tra il venditore ed il potenziale cliente. L'\textit{e-commerce} è come una relazione tra le parti, dove le alleanze, l'organizzazione e le comunicazioni sono rese possibili dalle nuove tecnologie.

\paragraph{Strategia di marketing}
\subparagraph{Supporto e impegno manageriale}
È richiesta la conoscenza delle potenzialità di Internet da parte del manager, che ha anche il compito di diffonderle in modo proattivo all'interno dell'azienda. L'impegno per il commercio via web aiuta a promuovere il suo sviluppo anche in altre aziende, ma richiede comunque supporto finanziario: c'è un'importante correlazione tra l'investimento effettuato ed il guadagno ottenuto. Per questo il coinvolgimento del livello amministrativo gioca il ruolo più critico.

\subparagraph{Obiettivi strategici}
Il successo dello sviluppo del web B2B dipendono da quanto chiaramente sono definiti gli obiettivi strategici per una organizzazione.

\subparagraph{Integrazione tra Internet e la strategia di marketing}
I responsabili marketing web di successo sono quelli che costruiscono un sistema in grado di integrarsi con le applicazioni esistenti e che possano offrire la formazione necessaria per il suo utilizzo. Sebbene Internet offre molti dei servizi, esso non rimpiazzerà i mezzi tradizionali: i clienti che comprano online continuano comunque a comprare attraverso altri mezzi. Pertanto, le aziende devono considerare il web-marketing come un complemento, piuttosto che un rimpiazzo. Molti clienti infatti preferiscono valutare un prodotto online e poi procedere con l'acquisto in altri modi, di persona o al telefono ad esempio.

\subparagraph{Definizione del target}
Definire chi tra clienti esterni, fornitori, venditori, rivenditori ed altri \textit{business partner} costituisce il target è una operazione primaria, in quanto determina come i canali, gli strumenti e il target stesso sono utilizzati e coinvolti.

\paragraph{Sito web}
\subparagraph{Design}
Un sito web ben progettato e con un bel design è il biglietto da visita dell'azienda nel web. La creazione di un portale richiede però un continuo sforzo per il suo aggiornamento e mantenimento, per continuare a soddisfare le aspettative degli utenti in base alle tendenze. Un ulteriore fattore è il contenuto, che deve essere di valore, accurato ed aggiornato, sia per attrarre nuovi clienti da ogni dove, sia per incoraggiarli a ritornare; le informazioni devono essere chiare e consistenti, in quanto i clienti fanno sempre una valutazione sulla loro utilità nel condurlo a prendere decisioni. La maggior parte degli utenti arriva sul web in cerca di informazioni, pertanto un portale che offre più dettagli sull'azienda e sui prodotti avrà sicuramente più successo. D'altro canto, le performance per l'accesso a tali informazioni sono un dato altrettanto importante: la progettazione del catalogo dei prodotti va sviluppata ancor prima dello stesso portale, in quanto determina la velocità di esecuzione delle \textit{query} di ricerca per migliaia di prodotti.

Vanno quindi considerate tutte le regole base di usabilità, tra cui la semplificazione della navigazione con \gls{breadcrumb} e \gls{tag}, l'interazione e la reattività verso i \gls{feedback} degli utenti, lo scambio di informazioni tra utenti e l'integrazione del web con altri canali di marketing.\virgolette{Web site must not only serve as an electronic order-taking interface but also have high value added informational content.}\autocite{b2bSuccessFactors}

\subparagraph{Promozione (offline ed online)}
\paragraph{Fattori globali}
\subparagraph{Comprensione dell'ambiente esterno}
\subparagraph{Risorse}
\subparagraph{Multilingua}
\subparagraph{Considerazioni culturali}
\subparagraph{Consegna internazionale}
\paragraph{Fattori interni}
\subparagraph{Infrastruttura tecnologica}
\subparagraph{Cultura interna}
\subparagraph{Forza vendita}
\subparagraph{Formazione}
\paragraph{Fattori esterni}
\subparagraph{Fiducia}
\subparagraph{Sicurezza}
\subparagraph{Relazioni}
\subparagraph{Infrastruttura di rete}
\subparagraph{Coinvolgimento del cliente}
\section{Il B2B di 4words}
Descrizione e analisi del B2B di Sanmarco Informatica.
\subsection{Struttura del progetto}
Come è strutturato il B2B, sia a livello funzionale, sia a livello progettuale
\subsection{Versionamento}
Come viene versionato
\subsection{Tecnologie utilizzate}
Analisi delle tecnologie utilizzate
\subsubsection{Back-end}
\subsubsection{Front-end}