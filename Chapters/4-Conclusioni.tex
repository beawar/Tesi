\chapter{Conclusioni}
\begin{flushright}
	\parbox{13cm}{\small In questo ultimo capitolo vengono tratte alcune conclusioni, sia per quanto riguarda gli obiettivi, sia a livello personale.}
\end{flushright}

\section{Resoconto delle attività}
Durante lo stage ho avuto la possibilità di analizzare il B2B sviluppato e venduto dall'azienda, quindi di progettare e implementare alcune personalizzazioni richieste dai clienti. Al periodo di formazione frontale iniziale sono seguite un paio di settimane in cui ho studiato la struttura e le procedure con cui il prodotto viene realizzato in azienda, applicando i concetti appresi nella settimana precedente a due casi reali, ovvero i progetti presentati nel capitolo precedente. Per fare questo, ho seguito il tutor aziendale nei suoi appuntamenti, svolgendo le attività che mi venivano di volta in volta assegnate. 

Inizialmente mi sono occupata di individuare e correggere alcune anomalie, svolgendo il ruolo di tester, per poi dedicarmi all'implementazione di nuove funzionalità, progettando il sistema, sviluppandolo e testandolo su ambienti dedicati. Uno di questi lavori è stato proprio il task per le newsletter, descritto in \hyperref[sec:task]{sezione \ref{sec:task}}. Al termine dello stage avevo quindi acquisito la capacità di analizzare le richieste, svilupparle e infine pubblicarle in ambiente di produzione del cliente.

In questa relazione ho elaborato l'analisi condotta nelle prime settimane, estendendo quanto emerso ai casi pratici a cui mi sono dedicata nel secondo periodo di stage.

\section{Considerazioni sul B2B}
Il B2B è dunque un'applicazione strettamente legata al gestionale offerto dall'azienda, al quale fa da supporto, ma che impone alcuni vincoli di progettazione. L'utilizzo dell'AS/400 come server costringe all'utilizzo del relativo database per la lettura dei dati, con tabelle difficilmente modificabili e un numero di campi limitato e non incrementabile. Sebbene siano stati predisposti tabelle e campi dedicati al portale, capita che per progettare nuove implementazioni si renda necessario scendere a compromessi con flag poco comprensibili per rappresentare variabili anche complesse. Per ovviare a questo problema in alcuni progetti è stato creato un database MySQL aggiuntivo, il quale, sebbene risolva i problemi appena citati, introduce di fatto complicazioni nella sincronizzazione e ridondanza dei dati.

Per quanto riguarda invece le tecnologie utilizzate, lo stretto legame con IBM ed i relativi prodotti convoglia la scelta verso tecnologie a volte poco conosciute e non sempre compatibili con gli strumenti utilizzati (RTC, lo strumento per il versionamento, è disponibile solo per Eclipse, pertanto non viene usato da coloro che sfruttano altri IDE, ad esempio per lo sviluppo in PHP, i quali devono affidarsi ad un altro strumento). La complessità del prodotto, la necessità di stabilità e un gruppo sottodimensionato rispetto al carico di lavoro non permette lo studio e l'utilizzo di nuove tecnologie o strumenti, che a prima vista possono risultare non adatti, ma che con un'attenta analisi potrebbero rivelare vantaggi importanti. Tuttavia, l'uso di Java come linguaggio primario permette di integrare velocemente nuove persone per lo sviluppo del portale, essendo uno dei linguaggi più insegnati a livello scolastico e accademico. In merito a JSF, Primefaces ed il Tomcat come contenitore web, essi sono relativamente semplici per un utilizzo base ed anche per questo continuano ad essere preferiti rispetto ad eventuali nuovi \textit{framework} emergenti.

Va considerato anche che il totale rinnovamento di un prodotto come il B2B richiede un impegno non indifferente, sia per gli sviluppatori, sia per l'azienda, che deve finanziare l'attività. Per effettuare un'operazione simile è quindi richiesto un profondo studio non solo dei nuovi linguaggi, ma anche del mercato, in modo tale da avere al termine del progetto un portale che soddisfi le esigenze emerse nel periodo di sviluppo.

Infine, una carenza abbastanza sentita all'interno del gruppo B2B è la mancanza di persone il cui compito principale è effettuare test prima dei rilasci. Il modo attuale di procedere nell'avanzamento dello sviluppo (sia a livello standard che per le personalizzazioni) si ripercuote nel prodotto finale, con frequenti aggiornamenti correttivi per anomalie anche banali, evitabili ed individuabili da tester dedicati. Questo genere di attività dipende però più dalle strategie aziendali che dai singoli programmatori, i quali non possono avere un adeguato successo nella verifica del loro stesso prodotto (sia esso il codice o il prodotto finale).

Nella sua versione attuale il B2B di 4words è quindi un prodotto valido e stabile, che si merita il successo che sta riscuotendo nel mercato come portale per la raccolta ordini, anche grazie ai vari moduli disponibili che ne permettono l'ampliamento e l'integrazione con diversi servizi.

\section{Raggiungimento degli obiettivi}
Prima di iniziare il periodo di stage sono stati definiti alcuni obiettivi, presentati nel piano di lavoro. Questi sono stati suddivisi in obbligatori, desiderabili ed opzionali. Al termine dello stage sono stati soddisfatti pienamente gli obiettivi obbligatori, riguardanti la realizzazione di nuove funzionalità per il B2B. Sono stati soddisfatti parzialmente gli obiettivi desiderabili, focalizzati più sulla conoscenza degli altri prodotti venduti dall'azienda: mentre alcuni concetti sono fondamentali per lo sviluppo del B2B, ce ne sono altri che non capita spesso di utilizzare, e che quindi non ho avuto modo di apprendere nel corso dello stage. Infine, gli obiettivi opzionali puntavano alla capacità di effettuare nuove analisi e stime per i clienti, una caratteristica acquisita solo in parte, sopratutto per quanto riguarda la stima di lavoro, che richiede un'ampia conoscenza delle funzionalità già disponibili.

La \hyperref[tab:obiettivi-2]{tabella \ref{tab:obiettivi-2}} rappresenta gli obiettivi del piano di lavoro aggiornati con gli esiti ottenuti.
\begin{table}
	\centering
	\begin{tabular}{|m{9.3cm}|c|}
		\hline
		\multicolumn{2}{|l|}{\textbf{Obbligatori}} \\
		\hline
		Progettazione dei portali Web per la raccolta degli ordini & Raggiunto \\
		\hline
		Realizzazione software Java & Raggiunto \\
		\hline
		Interazione con database SQL & Raggiunto \\
		\hline
		Realizzazione front-end HTML, CSS, JavaScript & Raggiunto \\
		\hline
		\multicolumn{2}{|l|}{\textbf{Desiderabili}}\\
		\hline
		Conoscenza dell'impianto commerciale del gestionale & Parzialmente raggiunto \\
		\hline
		Conoscenza dell'impianto amministrativo del gestionale & Parzialmente raggiunto \\
		\hline
		Autonomia della gestione con il cliente per raccolta nuove richieste & Non raggiunto \\
		\hline
		\multicolumn{2}{|l|}{\textbf{Opzionali}} \\
		\hline
		Analisi e stima nuove richieste clienti & Parzialmente raggiunto \\
		\hline
	\end{tabular}
	\caption{Raggiungimento degli obiettivi}
	\label{tab:obiettivi-2}
\end{table}

\section{Considerazioni sullo stage}
Complessivamente al termine dello stage mi ritengo soddisfatta di quanto svolto. In questo periodo ho potuto confrontarmi con la realtà aziendale, in cui non mi ero mai inserita prima. La possibilità di effettuare uno stage come conclusione del percorso di studio credo sia una delle migliori opportunità che l'Università, ed in particolare questo corso di studi, possa offrire ad uno studente, soprattutto per chi, come me, non ha mai avuto una vera occupazione prima. 

Le conoscenze acquisite durante i tre anni sono state ampiamente sufficienti per affrontare questo tipo di esperienza e mi hanno permesso di apprendere nuove tecnologie in breve tempo. Inoltre, ho potuto conoscere ed imparare alcuni concetti fondamentali nell'ambito economico-aziendale che mi saranno sicuramente utili per le esperienze future. La tipologia di stage, con frequenti interazioni con il cliente, mi ha infine permesso di crescere anche a livello personale.
