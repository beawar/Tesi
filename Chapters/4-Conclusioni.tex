\chapter{Conclusioni}
\begin{flushright}
	\parbox{13cm}{\small In questo ultimo capitolo vengono tratte alcune conclusioni, sia per quanto riguarda gli obiettivi, sia a livello personale.}
\end{flushright}
\section{Raggiungimento degli obiettivi}
Prima di iniziare il periodo di stage sono stati definiti alcuni obiettivi, presentati nel piano di lavoro. Questi sono stati suddivisi in obbligatori, desiderabili ed opzionali. Al termine dello stage sono stati soddisfatti pienamente gli obiettivi obbligatori, riguardanti la realizzazione di nuove funzionalità per il B2B. Sono stati soddisfatti parzialmente gli obiettivi desiderabili, focalizzati più sulla conoscenza degli altri prodotti venduti dall'azienda: mentre alcuni concetti sono fondamentali per lo sviluppo del B2B, ce ne sono altri che non capita spesso di utilizzare, e che quindi non ho avuto modo di apprendere nel corso dello stage. Infine, gli obiettivi opzionali puntavano alla capacità di effettuare nuove analisi e stime per i clienti, una caratteristica acquisita solo in parte, sopratutto per quanto riguarda la stima di lavoro, che richiede un'ampia conoscenza delle funzionalità già di disponibili.

La \hyperref[tab:obiettivi-2]{tabella \ref{tab:obiettivi-2}} rappresenta gli obiettivi del piano di lavoro aggiornati con gli esiti ottenuti.
\begin{table}
	\centering
	\begin{tabular}{|m{9.3cm}|c|}
		\hline
		\multicolumn{2}{|l|}{\textbf{Obbligatori}} \\
		\hline
		Progettazione dei portali Web per la raccolta degli ordini & Raggiunto \\
		\hline
		Realizzazione software Java & Raggiunto \\
		\hline
		Interazione con database SQL & Raggiunto \\
		\hline
		Realizzazione front-end HTML, CSS, JavaScript & Raggiunto \\
		\hline
		\multicolumn{2}{|l|}{\textbf{Desiderabili}}\\
		\hline
		Conoscenza dell'impianto commerciale del gestionale & Parzialmente raggiunto \\
		\hline
		Conoscenza dell'impianto amministrativo del gestionale & Parzialmente raggiunto \\
		\hline
		Autonomia della gestione con il cliente per raccolta nuove richieste & Non raggiunto \\
		\hline
		\multicolumn{2}{|l|}{\textbf{Opzionali}} \\
		\hline
		Analisi e stima nuove richieste clienti & Parzialmente raggiunto \\
		\hline
	\end{tabular}
	\caption{Raggiungimento degli obiettivi}
	\label{tab:obiettivi-2}
\end{table}

\section{Considerazioni personali}
Complessivamente al termine dello stage mi ritengo soddisfatta di quanto svolto. In questo periodo ho potuto confrontarmi con la realtà aziendale, in cui non mi ero mai inserita prima. La possibilità di effettuare uno stage come conclusione del percorso di studio credo sia una delle migliori opportunità che l'Università, ed in particolare questo corso di studi, possa offrire ad uno studente, soprattutto per chi, come me, non ha mai avuto una vera occupazione prima. 

Le conoscenze acquisite durante i tre anni sono state ampiamente sufficienti per affrontare questo tipo di esperienza e mi hanno permesso di apprendere nuove tecnologie in breve tempo. Inoltre, ho potuto conoscere ed imparare alcuni concetti fondamentali nell'ambito economico-aziendale che mi saranno sicuramente utili per le esperienze future. La tipologia di stage, con frequenti interazioni con il cliente, mi ha infine permesso di crescere anche a livello personale.
