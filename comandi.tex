\documentclass[
11pt, % The default document font size, options: 10pt, 11pt, 12pt
%oneside, % Two side (alternating margins) for binding by default, uncomment to switch to one side
openright,
italian, % ngerman for German
singlespacing, % Single line spacing, alternatives: onehalfspacing or doublespacing
%draft, % Uncomment to enable draft mode (no pictures, no links, overfull hboxes indicated)
%nolistspacing, % If the document is onehalfspacing or doublespacing, uncomment this to set spacing in lists to single
%liststotoc, % Uncomment to add the list of figures/tables/etc to the table of contents
%toctotoc, % Uncomment to add the main table of contents to the table of contents
parskip, % Uncomment to add space between paragraphs
%nohyperref, % Uncomment to not load the hyperref package
headsepline, % Uncomment to get a line under the header
%chapterinoneline, % Uncomment to place the chapter title next to the number on one line
%consistentlayout, % Uncomment to change the layout of the declaration, abstract and acknowledgements pages to match the default layout
]{MastersDoctoralThesis} % The class file specifying thse document structure

\usepackage[utf8]{inputenc} % Required for inputting international characters
\usepackage[T1]{fontenc} % Output font encoding for international characters

\usepackage{palatino} % Use the Palatino font by default

\usepackage[backend=bibtex,style=authoryear,natbib=true]{biblatex} % Use the bibtex backend with the authoryear citation style (which resembles APA)

% Bibliografia
\addbibresource{bibliografia.bib} % The filename of the bibliography
\usepackage[autostyle=true]{csquotes} % Required to generate language-dependent quotes in the bibliography

\newcommand{\virgolette}[1]{``#1''}
% Per l'allineamento delle immagini
\usepackage[export]{adjustbox}

% necessario per risolvere il problema del carattere invisibile per l'a capo
\DeclareUnicodeCharacter{00A0}{ }

% elenca anche i paragrafi nell'indice
\setcounter{tocdepth}{6}

% permetti di definire dei colori
\usepackage{xcolor}

% permette di inserire le immagini/tabelle esattamente dove viene usato il
% comando \begin{figure}[H] ... \end{figure}
% evitando che venga spostato in automatico
\usepackage{float}

% permette l'inserimento di url e di altri tipi di collegamento
\usepackage[colorlinks=true]{hyperref}

\hypersetup{
	colorlinks=true, % false: boxed links; true: colored links
	citecolor=black,
	filecolor=black,
	linkcolor=RoyalBlue, % color of internal links
	urlcolor=Maroon  % color of external links
}


% permette al comando \url{...} di andare a capo a metà di un link
\usepackage{breakurl}

% immagini
\usepackage{graphicx}

% permette di riferirsi all'ultima pagina con \pageref{LastPage}
\usepackage{lastpage}

% tabelle su più pagine
\usepackage{longtable}

% per avere dei comandi in più da poter usare sulle tabelle
\usepackage{booktabs}

% tabelle con il campo X per riempire lo spazio rimanente sulla riga
\usepackage{tabularx}

% multirow per tabelle
\usepackage{multirow}

% permette di fare longtable larghe tutta la pagina (parametro x)
\usepackage{tabu}

% imposta lo spazio tra le righe di una tabella
\setlength{\tabulinesep}{6pt}

% personalizza l'intestazione e piè di pagina
%\usepackage{fancyhdr}

% permette di inserire caratteri speciali
\usepackage{textcomp}

% permette di aggiustare i margini e centrare tabelle e figure
\usepackage{changepage}

% Per numerare le tabelle e le figure con la sezione in cui si trovano 
\usepackage{amsmath}
\numberwithin{figure}{section}
\numberwithin{table}{section}

% Use \ul{arg} to undlerline text
\usepackage{soul}

% Per le pagine in orientamento orizzontale
\usepackage{pdflscape}
\usepackage{afterpage}

% Set vertical space in tables
\def\arraystretch{1.5}

% glossary
\usepackage[xindy, acronym]{glossaries}

\definecolor{darkblue}{RGB}{15, 83, 193}
\renewcommand*{\glstextformat}[1]{\textcolor{darkblue}{#1}}

\oddsidemargin=30pt \evensidemargin=20pt%impostano i margini


%----------------------------------------------------------------------------------------
%	THESIS INFORMATION
%----------------------------------------------------------------------------------------

\thesistitle{Analisi e personalizzazione di portali B2B} % Your thesis title, this is used in the title and abstract, print it elsewhere with \ttitle
\supervisor{Prof.ssa Ombretta Gaggi} % Your supervisor's name, this is used in the title page, print it elsewhere with \supname
\examiner{} % Your examiner's name, this is not currently used anywhere in the template, print it elsewhere with \examname
\degree{Laurea in Informatica} % Your degree name, this is used in the title page and abstract, print it elsewhere with \degreename
\author{Beatrice Guerra} % Your name, this is used in the title page and abstract, print it elsewhere with \authorname
\addresses{} % Your address, this is not currently used anywhere in the template, print it elsewhere with \addressname

\subject{} % Your subject area, this is not currently used anywhere in the template, print it elsewhere with \subjectname
\keywords{} % Keywords for your thesis, this is not currently used anywhere in the template, print it elsewhere with \keywordnames
\university{Università degli Studi di Padova} % Your university's name and URL, this is used in the title page and abstract, print it elsewhere with \univname
\department{Dipartimento di Matematica \virgolette{Tullio Levi-Civita}} % Your department's name and URL, this is used in the title page and abstract, print it elsewhere with \deptname
\faculty{Corso di Laurea in Informatica} % Your faculty's name and URL, this is used in the title page and abstract, print it elsewhere with \facname

\AtBeginDocument{
	\hypersetup{pdftitle=\ttitle} % Set the PDF's title to your title
	\hypersetup{pdfauthor=\authorname} % Set the PDF's author to your name
	\hypersetup{pdfkeywords=\keywordnames} % Set the PDF's keywords to your keywords
}