%%%%%%%%%%%%%%%%%%%%%%%%%%%%%%%%%%%%%%%%%%%%%%%%%
%                ACRONIMI
%%%%%%%%%%%%%%%%%%%%%%%%%%%%%%%%%%%%%%%%%%%%%%%%%

\newglossaryentry{IBM}{
	type=\acronymtype, 
	name={IBM}, 
	description={International Business Machines Corporation}, 
	first={IBM},
	see=[Glossary:]{ibmg}
}

\newglossaryentry{ERP}{
	type=\acronymtype, 
	name={ERP}, 
	description={Enterprise Resource Planning}, 
	first={Enterprise Resource Planning (ERP)}, 
	see=[Glossary:]{erpg}
}

\newglossaryentry{IOS}{
	type=\acronymtype, 
	name={iOS}, 
	description={iPhone Operating System}, 
	first={iOS \glsadd{iosg}}, 
	see=[Glossary:]{iosg}
}

\newglossaryentry{B2B}{
	type=\acronymtype, 
	name={B2B}, 
	description={Business-to-Business}, 
	first={B2B \glsadd{b2bg}}, 
	see=[Glossary:]{b2bg}
}

\newglossaryentry{B2C}{
	type=\acronymtype, 
	name={B2C}, 
	description={Business-to-Customer}, 
	first={B2B \glsadd{b2cg}}, 
	see=[Glossary:]{b2cg}
}

\newglossaryentry{CI}{
	type=\acronymtype, 
	name={CI}, 
	description={Customer Intelligence}, 
	first={CI (Customer Intelligence) \glsadd{cig}}, 
	see=[Glossary:]{cig}
}

\newglossaryentry{BI}{
	type=\acronymtype, 
	name={BI}, 
	description={Business Intelligence}, 
	first={BI (Business Intelligence) \glsadd{big}}, 
	see=[Glossary:]{big}
}

\newglossaryentry{CPM}{
	type=\acronymtype, 
	name={CPM}, 
	description={Corporate Performance Management}, 
	first={CPM (Corporate Performance Management) \glsadd{cpmg}}, 
	see=[Glossary:]{cpmg}
}

\newglossaryentry{SEO}{
	type=\acronymtype, 
	name={SEO}, 
	description={Search Engine Optimization}, 
	first={SEO \glsadd{seog}}, 
	see=[Glossary:]{seog}
}

\newglossaryentry{PPC}{
	type=\acronymtype, 
	name={PPC}, 
	description={Pay Per Click}, 
	first={PPC (Pay Per Click) \glsadd{ppcg}}, 
	see=[Glossary:]{ppcg}
}

\newglossaryentry{PMI}{
	type=\acronymtype, 
	name={PMI}, 
	description={Piccole e Medie Imprese}, 
	first={PMI (Piccole e Medie Imprese)}
}

\newglossaryentry{JSF}{
	type=\acronymtype, 
	name={JSF}, 
	description={JavaServer Faces}, 
	first={JSF (JavaServer Faces)},
	see=[Glossary:]{jsfg}
}

\newglossaryentry{CSS}{
	type=\acronymtype, 
	name={CSS}, 
	description={Cascading Style Sheets}, 
	first={CSS}
}

\newglossaryentry{IDE}{
	type=\acronymtype, 
	name={IDE}, 
	description={Integrated Development Environment}, 
	first={IDE},
	see=[Glossary:]{ideg}
}

\newglossaryentry{IVA}{
	type=\acronymtype, 
	name={IVA}, 
	description={Imposta sul Valore Aggiunto}, 
	first={IVA},
	see=[Glossary:]{ivag}
}

\newglossaryentry{API}{
	type=\acronymtype, 
	name={API}, 
	description={Application Programming Interface}, 
	first={API},
	see=[Glossary:]{apig}
}

\newglossaryentry{REST}{
	type=\acronymtype, 
	name={REST}, 
	description={REpresentational State Transfer}, 
	first={REST},
	see=[Glossary:]{restg}
}

\newglossaryentry{SOAP}{
	type=\acronymtype, 
	name={SOAP}, 
	description={Simple Object Access Protocol}, 
	first={SOAP},
	see=[Glossary:]{soapg}
}

\newglossaryentry{CMS}{
	type=\acronymtype, 
	name={CMS}, 
	description={Content Management System}, 
	first={CMS},
	see=[Glossary:]{cmsg}
}

\newglossaryentry{MVC}{
	type=\acronymtype, 
	name={MVC}, 
	description={Model View Controller}, 
	first={MVC (Model View Controller)},
	see=[Glossary:]{mvcg}
}

\newglossaryentry{XHTML}{
	type=\acronymtype, 
	name={XHTML}, 
	description={eXtensible HyperText Markup Language}, 
	first={XHTML},
	see=[Glossary:]{xhtmlg}
}

\newglossaryentry{RTC}{
	type=\acronymtype,
	name={RTC},
	description={Rational Team Concert},
	first={RTC (Rational Team Concert)},
	see=[Glossary:]{rtcg}
}

\newglossaryentry{JSP}{
	type=\acronymtype,
	name={JSP},
	description={JavaServer Pages},
	first={JSP (JavaServer Pages)},
	see=[Glossary:]{jspg}
}

\newglossaryentry{DAO}{
	type=\acronymtype,
	name={DAO},
	description={Data Access Object},
	first={DAO (Data Access Object)},
	see=[Glossary:]{daog}
}

\newglossaryentry{el}{
	type=\acronymtype,
	name={EL},
	description={Expression Language},
	first={Expression Language (EL)},
	see=[Glossary:]{elg}
}

\newglossaryentry{EJB}{
	type=\acronymtype,
	name={EJB},
	description={Enterprise JavaBeans},
	first={Enterprise JavaBeans (EJB)},
	see=[Glossary:]{ejbg}
}

\newglossaryentry{URL}{
	type=\acronymtype,
	name={URL},
	description={Uniform Resource Locator},
	first={URL},
	see=[Glossary:]{urlg}
}

\newglossaryentry{CED}{
	type=\acronymtype,
	name={CED},
	description={Centro Elaborazione Dati},
	first={CED},
	see=[Glossary:]{cedg}
}

%%%%%%%%%%%%%%%%%%%%%%%%%%%%%%%%%%%%%%%%%%%%%%%%%
%                GLOSSARIO
%%%%%%%%%%%%%%%%%%%%%%%%%%%%%%%%%%%%%%%%%%%%%%%%%

\newglossaryentry{software house}{
	name={software house},
	description={Azienda specializzata principalmente nella produzione di software e applicazioni}
}

\newglossaryentry{ibmg}{
	name={IBM},
	description={Azienda statunitense, tra le maggiori al mondo nel settore informatico. Produce e commercializza hardware e software e servizi informatici, offre infrastrutture, servizi di hosting, servizi di cloud computing e consulenza. Oggi IBM sta emergendo come una società che fornisce soluzioni cognitive e piattaforme cloud}
}

\newglossaryentry{erpg}{
	name={enterprise resource planning},
	description={Enterprise resource planning (letteralmente \virgolette{pianificazione delle risorse d'impresa}, spesso abbreviato in ERP) è un sistema di gestione, chiamato in informatica sistema informativo, che integra tutti i processi di business rilevanti di un'azienda (vendite, acquisti, gestione magazzino, contabilità etc.)}
}

\newglossaryentry{iosg}{
	name={iOS},
	description={Sistema operativo sviluppato da Apple per iPhone, iPod touch e iPad}
}

\newglossaryentry{android}{
	name={Android},
	description={Sistema operativo per dispositivi mobili sviluppato da Google Inc. e basato sul kernel Linux; non è però da considerarsi propriamente un sistema unix-like o una distribuzione GNU/Linux, dato che la quasi totalità delle utilità GNU è sostituita da software in Java}
}

\newglossaryentry{b2bg}{
	name={business-to-business},
	description={Business-to-business, spesso indicato con l'acronimo B2B, in italiano commercio interaziendale, è una locuzione utilizzata per descrivere le transazioni commerciali elettroniche tra imprese, distinguendole da quelle che intercorrono tra le imprese e altri gruppi, come quelle tra una ditta e i consumatori/clienti individuali (B2C, dall'inglese Business to Customer o Business to Consumer, in italiano vendita al dettaglio) oppure quelle tra una impresa e il governo (B2G, dall'inglese Business to Government, lett. "azienda-verso-governo)}
	see=[Glossary:]{b2cg}
}

\newglossaryentry{b2cg}{
	name={business-to-customer},
	description={Con Business-to-customer o Business-to-consumer, spesso abbreviato in B2C, si indicano le relazioni che un'impresa commerciale detiene con i suoi clienti per le attività di vendita e/o di assistenza. Questa sigla è utilizzata soprattutto quando l'interazione tra impresa e cliente avviene tramite internet, ovvero nel caso del commercio elettronico}
}

\newglossaryentry{cig}{
	name={Customer Intelligence},
	description={Il processo di recupero e analisi di informazioni riguardanti i clienti: i loro dettagli e le loro attività, con l'obiettivo di costruire relazioni con il cliente profonde ed solide e di migliorare la strategia di marketing}
}

\newglossaryentry{big}{
	name={Business Intelligence},
	description={Con la locuzione business intelligence (BI) ci si può solitamente riferire a un insieme di processi aziendali per raccogliere dati ed analizzare informazioni strategiche, la tecnologia utilizzata per realizzare questi processi e le informazioni ottenute come risultato di questi processi}
}

\newglossaryentry{cpmg}{
	name={Corporate Performance Management},
	description={Con Corporate Performance Management (CPM) si intende  un insieme di processi per la gestione, la misurazione e il controllo delle performance aziendali, a seguito dell'idinteficazione degli obiettivi da raggiungere in un dato periodo. In tal senso è da considerarsi sinonimo di Business Performance Management (BPM) e Enterprise Performance Management (EPM)}
}

\newglossaryentry{cybersecurity}{
	name={cybersecurity},
	description={Sottoclasse del più ampio concetto di information security. Per cybersecurity si intende infatti quell'ambito dell'information security prettamente ed esclusivamente dipendente dalla tecnologia informatica. Nell'utilizzare il termine cybersecurity si vuole intendere, in particolare, un approccio mirato ad enfatizzare non tanto le misure di prevenzione (ovvero quelle misure che agiscono riducendo la probabilità di accadimento di una minaccia), ma soprattutto le misure di protezione (ovvero quelle misure che agiscono riducendo la gravità del danno realizzato da una minaccia)}
}

\newglossaryentry{Open Power Foundation IBM}{
	name={Open Power Foundation IBM},
	description={La Open Power Foundation è una collaborazione attorno ai prodotti Power Architecture iniziata da IBM e annunciata come l'\virgolette{Open Power Consortium}. L'obiettivo è di permettere ai venditori di ambienti server di costruire server, reti e storage hardware personalizzati per futuri data centers e cloud computing}
}

\newglossaryentry{Beacon Award}{
	name={Beacon Award},
	description={Il programma IBM Beacon Awards premia i Business Partner IBM che offrono soluzioni straordinarie che consentono di ottenere il valore aziendale e trasformare il modo di agire di clienti e settori. I premi del 2017 riconosceranno i risultati conseguiti in una serie di aree di soluzioni – inclusi IBM Watson, Cloud e Analytics – per condurre i clienti nell'era cognitiva}
}

\newglossaryentry{Power}{
	name={Power},
	description={IBM Power Systems è la linea di server basati su tecnologie open e pensati per le applicazioni mission-critical}
}

\newglossaryentry{seog}{
	name={Search Engine Optimization},
	description={Con il termine ottimizzazione per i motori di ricerca (in lingua inglese Search Engine Optimization, in acronimo SEO) si intendono, nel linguaggio di internet, tutte quelle attività volte a migliorare la visibilità di un sito web sui motori di ricerca al fine di migliorare (o mantenere) il posizionamento nelle pagine di risposta alle interrogazioni degli utenti del web. A sua volta, il buon posizionamento di un sito web nelle pagine di risposta dei motori di ricerca è funzionale alla visibilità dei prodotti/servizi venduti}
}

\newglossaryentry{ppcg}{
	name={Pay Per Click},
	description={Il pay per click (PPC) è una modalità di acquisto e pagamento della pubblicità online; l'inserzionista paga una tariffa unitaria in proporzione ai click (click-through rate), ovvero solo quando un utente clicca effettivamente sull'annuncio pubblicitario. Un esempio di pubblicità pay per click è rappresentato dal keyword advertising, cioè annunci sponsorizzati che compaiono a lato dei risultati "puri" dei motori di ricerca}
}

\newglossaryentry{stakeholder}{
	name={Stakeholder},
	description={con il termine stakeholder (o portatore di interesse) si indica genericamente un soggetto (o un gruppo di soggetti) influente nei confronti di un'iniziativa economica, che sia un'azienda o un progetto.
	Fanno, ad esempio, parte di questo insieme: i clienti, i fornitori, i finanziatori come banche e azionisti (o shareholder), i collaboratori, dipendenti ma anche gruppi di interesse locali o gruppi di interesse esterni, come i residenti di aree limitrofe all'azienda e le istituzioni statali relative all'amministrazione locale}
}

\newglossaryentry{jsfg}{
	name={JavaServer Faces},
	description={JavaServer Faces (JSF) è una tecnologia Java, basata sul design pattern architetturale Model-View-Controller (MVC), il cui scopo è quello di semplificare lo sviluppo dell'interfaccia utente (UI) di una applicazione Web; può quindi essere considerata un framework per componenti lato server di interfaccia utente}
}

\newglossaryentry{less}{
	name={Less},
	description={Less è un preprocessore CSS che estende il normale linguaggio CSS permettendo (oltre alla normale sintassi dei fogli di stile) anche l'utilizzo di funzioni, operatori e variabili, la nidificazione delle istruzioni, la creazione di "mixin" e numerose altre caratteristiche che rendono il codice più facile da scrivere, da manutenere e da comprendere}
}

\newglossaryentry{primefaces}{
	name={Primefaces},
	description={PrimeFaces è una suite di componenti open source per la realizzazione dell'interfaccia utente di applicazioni web basate sulla tecnologia Java Server Faces. È sviluppata dalla PrimeTek}
}

\newglossaryentry{eclipse}{
	name={Eclipse},
	description={PrimeFaces è una suite di componenti open source per la realizzazione dell'interfaccia utente di applicazioni web basate sulla tecnologia Java Server Faces. È sviluppata dalla PrimeTek}
}

\newglossaryentry{ideg}{
	name={Integrated Development Environment},
	description={(Ambiente di sviluppo integrato) Strumento software che consiste di più componenti, da cui appunto il nome integrato: un editor di codice sorgente; un compilatore e/o un interprete; un tool di building automatico; un debugger. A volte è integrato anche con un sistema di controllo di versione e uno o più tool per semplificare la costruzione di una GUI}
}

\newglossaryentry{ivag}{
	name={Imposta sul Valore Aggiunto},
	description={Imposta adottata in sessantotto Paesi del mondo (tra i quali anche vari membri dell'UE) applicata sul valore aggiunto di ogni fase della produzione, di scambio di beni e servizi}
}

\newglossaryentry{breadcrumb}{
	name={breadcrumb},
	description={Tecnica di navigazione usata nelle interfacce utente. Il loro scopo è quello di fornire agli utenti un modo di tener traccia della loro posizione in documenti o programmi}
}

\newglossaryentry{tag}{
	name={tag},
	description={Un tag (cioè etichetta, marcatore, identificatore) è una parola chiave o un termine associato a un'informazione (un'immagine, una mappa geografica, un post, un video clip, ecc.), che descrive l'oggetto rendendo possibile la classificazione e la ricerca di informazioni basata su parole chiave}
}

\newglossaryentry{feedback}{
	name={feedback},
	description={Giudizio contenente possibili migliorie e segnalazioni di errori, inviato allo sviluppatore di un'applicazione da un utente che la collauda}
}

\newglossaryentry{wizard}{
	name={wizard},
	description={Procedura informatica, generalmente inglobata in una applicazione più complessa, che permette all'utente di eseguire determinate operazioni (solitamente complesse) tramite una serie di passi successivi}
}

\newglossaryentry{apig}{
	name={API},
	description={Insieme di procedure disponibili al programmatore, di solito raggruppate a formare un set di strumenti specifici per l'espletamento di un determinato compito all'interno di un certo programma. Spesso con tale termine si intendono le librerie software disponibili in un certo linguaggio di programmazione}
}

\newglossaryentry{restg}{
	name={REST},
	description={Architettura software per i sistemi di ipertesto distribuiti come il \textit{World Wide Web}. Il termine REST è spesso usato nel senso di descrivere ogni semplice interfaccia che trasmetta dati su HTTP senza un livello opzionale come \Gls{SOAP}}
}

\newglossaryentry{Magento}{
	name={Magento},
	description={CMS open source per l'e-commerce lanciato il 31 marzo 2008}
}

\newglossaryentry{Prestashop}{
	name={Prestashop},
	description={CMS open source utilizzato per realizzare siti di e-commerce. Nasce nel 2007 e, a differenza dei CMS più \virgolette{generici} diffusi all'epoca della sua prima release (WordPress e Joomla!), Prestashop è interamente pensato per lo sviluppo e la gestione dell'e-commerce.}
}

\newglossaryentry{soapg}{
	name={SOAP},
	description={Protocollo per lo scambio di messaggi tra componenti software, tipicamente nella forma di componentistica software. La parola \textit{object} in \textit{Simple Object Access Protocol} manifesta che l'uso del protocollo dovrebbe effettuarsi secondo il paradigma della programmazione orientata agli oggetti}
}

\newglossaryentry{cmsg}{
	name={Content Management System},
	description={Strumento software, installato su un server web, il cui compito è facilitare la gestione dei contenuti di siti web, svincolando il webmaster da conoscenze tecniche specifiche di programmazione web}
}

\newglossaryentry{mvcg}{
	name={Model View Controller},
	description={Pattern architetturale molto diffuso nello sviluppo di sistemi software, in particolare nell'ambito della programmazione orientata agli oggetti, in grado di separare la logica di presentazione dei dati dalla logica di business}
}

\newglossaryentry{RPG}{
	name={RPG},
	description={Linguaggio di programmazione nativo per minicomputer IBM della serie iSeries, denominata anche, più comunemente, AS/400}
}

\newglossaryentry{AS400}{
	name={AS/400},
	description={Minicomputer sviluppato dall'IBM per usi prevalentemente aziendali, come supporto del sistema informativo gestionale. Attualmente viene utilizzato soprattutto nelle moderne reti di computer come server di applicazioni software, tipicamente di tipo gestionale o comunque di business management, o server di rete (internet/intranet)}
}

\newglossaryentry{xhtmlg}{
	name={XHTML},
	description={Linguaggio di murkup che associa alcune proprietà dell'XML con le caratteristiche dell'HTML: un file XHTML è un pagina HTML scritta in conformità con lo standard XML}
}

\newglossaryentry{framework}{
	name={framework},
	description={Architettura logica di supporto (spesso un'implementazione logica di un particolare design pattern) su cui un software può essere progettato e realizzato, spesso facilitandone lo sviluppo da parte del programmatore}
}

\newglossaryentry{Spring Core}{
	name={Spring Core},
	description={Framework open source per lo sviluppo di applicazioni su piattaforma Java. Esistono numerose estensioni per la costruzione di applicazioni web-based costruite sul modello della piattaforma Java EE}
}

\newglossaryentry{rtcg}{
	name={RTC},
	description={RTC (Rational Team Concert) è uno strumento di collaborazione per lo sviluppo software realizzato da Rational Software, una branca di IBM. Fornisce un ambiente collaborativo che i team di sviluppo utilizzano per gestire tutti gli aspetti legati al loro lavoro (tra cui la pianificazione, i task, il controllo di versione, la gestione della configurazione e i report)}
}

\newglossaryentry{Git}{
	name={Git},
	description={Sistema di controllo di versione per il tracciamento dei cambiamenti nei file e per il coordinamento del lavoro di più persone sugli stessi}
}

\newglossaryentry{Jenkins}{
	name={Jenkins},
	description={Server di automatizzazione open source, aiuta ad automatizzare parti del processo di sviluppo del software tramite integrazione continua}
}

\newglossaryentry{Maven}{
	name={Maven},
	description={Strumento per la compilazione automatizzata usato principalmente per i progetti Java}
}

\newglossaryentry{JavaEE}{
	name={JavaEE},
	description={Enterprise Edition della Java Platform, anche nota come J2EE (Java 2 Enterprise Edition), è una specifica le cui implementazioni vengono principalmente sviluppate in linguaggio di programmazione Java e ampiamente utilizzata nella programmazione Web},
	first={JavaEE (Enterprise Edition)}
}

\newglossaryentry{servlet}{
	name={servlet},
	description={Oggetti scritti in linguaggio Java che operano all'interno di un server web}
}

\newglossaryentry{jspg}{
	name={JavaServer Pages},
	description={Tecnologia di programmazione Web in Java per lo sviluppo della logica di presentazione (tipicamente secondo il pattern MVC) di applicazioni Web, fornendo contenuti dinamici in formato HTML o XML}
}

\newglossaryentry{daog}{
	name={Data Access Object},
	description={Pattern architetturale per la gestione della persistenza: si tratta fondamentalmente di una classe che rappresenta un'entità tabellare di un database relazionale, usata principalmente in applicazioni web per stratificare e isolare l'accesso al data layer da parte della business logic, creando un maggiore livello di astrazione ed una più facile manutenibilità}
}

\newglossaryentry{facelets}{
	name={facelets},
	description={Sistema di template per il web open source, tramite costrutti XML. È la tecnologia utilizzata per creare l'interfaccia utente in JSF}
}

\newglossaryentry{IoC}{
	name={Inversion of Control},
	description={Pattern per cui un componente di livello applicativo riceve il controllo da un componente appartenente a un libreria riusabile. Questo schema ribalta quello tradizionale della programmazione procedurale, dove il codice applicativo svolge i propri compiti richiamando (e quindi passando il controllo a) procedure di libreria},
	first={Inversion of Control (IoC)}
}

\newglossaryentry{elg}{
	name={Expression Language},
	description={Linguaggio di scripting che permette di connettere le pagine JSF al back-end Java}
}

\newglossaryentry{ejbg}{
	name={Enterprise JavaBeans},
	description={Componenti che implementano la logica di business in un'applicazione JavaEE}
}

\newglossaryentry{descrittore di distribuzione}{
	name={descrittore di distribuzione},
	description={File XML che specifica opzioni di configurazione e di contenitore per un'applicazione o un modulo}
}

\newglossaryentry{urlg}{
	name={URL},
	description={Sequenza di caratteri che identifica univocamente l'indirizzo di una risorsa in Internet, tipicamente presente su un host server, come ad esempio un documento, un'immagine o un video, rendendola accessibile ad un client}
}

\newglossaryentry{cedg}{
	name={CED},
	description={L'unità organizzativa che coordina e mantiene le apparecchiature ed i servizi di gestione dei dati, ovvero l'infrastruttura IT a servizio di una o più aziende. In alcune realtà può essere denominato "Servizio elaborazione dati" (SED)}
}






